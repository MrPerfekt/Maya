\chapter{Zusammenfassung}
\label{chap:zusammenfassung}

In dieser Arbeit wurde ein Prototyp vorgestellt, der in Zukunft in der Lehrveranstaltung "`Mikrocomputer Labor"' verwendet werden soll.
Die Aufgaben, welche von den Studierenden gelöst werden müssen, sind Teilaufgaben der Regelung einer Spiegelablenkeinheit auf eine Scheibe mit einem Detektor.

Bei Arbeiten, welche einen repräsentativen Charakter haben, gibt es neben den technischen Eigenschaften auch gewisse Voraussetzungen, welche das Design der Arbeit betreffen.
Dadurch treten Probleme auf, mit denen normalerweise nicht gerechnet werden können.
Das am Anfang geplante Pendel war technisch gesehen voll funktionsfähig, jedoch war dieses optisch nicht ansprechend.
Unregelmäßige Schwingeigenschaften und eine zu instabile Konstruktionen waren die größte Probleme, weshalb das Pendel durch eine Scheibe ersetzt wurde.

Viele Schwierigkeiten dieser Arbeit sind bei der Erkennung des Lasers gelegen.
Dafür musste nicht nur ein geeignetes Verfahren gesucht werden, sondern auch die Aufbereitung der so gewonnenen Informationen ist ein interessanter Aufgabenbereich.
In diesem Teil der Arbeit gibt es durchaus noch Herausforderungen wie das Implementieren eines PSSG oder eines PUSD.

Eine weiterer interessanter Themenbereich, welcher in dieser Arbeit nur kurz angeschnitten wurde, ist die verwendete Regelung.
Der Übergang von einem Dreipunktregler zu einem Zustandsregler oder anderen Verfahren, würde dabei eine merkbare Verbesserung der Regeleigenschaften versprechen.

%Wichtigsten Ergebnisse / Was soll sich Leser merken?
%Unvorhergesehene Schwierigkeiten / Lösungen
%-Pendel
%Offene Fragen für weiteren Arbeit
%-PSSG / PUSD
%-Verbesserter Regelalgorithmus
